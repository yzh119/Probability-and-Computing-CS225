%!TEX program = xelatex
%%%%%%%%%%%%%%%%%%%%%%%%%%%%%%%%%%%%%%%%%
% Short Sectioned Assignment
% LaTeX Template
% Version 1.0 (5/5/12)
%
% This template has been downloaded from:
% http://www.LaTeXTemplates.com
%
% Original author:
% Frits Wenneker (http://www.howtotex.com)
%
% License:
% CC BY-NC-SA 3.0 (http://creativecommons.org/licenses/by-nc-sa/3.0/)
%
%%%%%%%%%%%%%%%%%%%%%%%%%%%%%%%%%%%%%%%%%

%----------------------------------------------------------------------------------------
%	PACKAGES AND OTHER DOCUMENT CONFIGURATIONS
%----------------------------------------------------------------------------------------

\documentclass[paper=a4, fontsize=11pt]{scrartcl} % A4 paper and 11pt font size
\usepackage{xeCJK}
\usepackage[T1]{fontenc} % Use 8-bit encoding that has 256 glyphs
\usepackage{fourier} % Use the Adobe Utopia font for the document - comment this line to return to the LaTeX default
\usepackage[english]{babel} % English language/hyphenation
\usepackage{amsmath,amsfonts,amsthm} % Math packages

\usepackage{lipsum} % Used for inserting dummy 'Lorem ipsum' text into the template
\usepackage{enumerate}
\usepackage{sectsty} % Allows customizing section commands
\allsectionsfont{\normalfont\scshape} % Make all sections centered, the default font and small caps

\usepackage{fancyhdr} % Custom headers and footers
\pagestyle{fancyplain} % Makes all pages in the document conform to the custom headers and footers
\fancyhead{} % No page header - if you want one, create it in the same way as the footers below
\fancyfoot[L]{} % Empty left footer
\fancyfoot[C]{} % Empty center footer
\fancyfoot[R]{\thepage} % Page numbering for right footer
\renewcommand{\headrulewidth}{0pt} % Remove header underlines
\renewcommand{\footrulewidth}{0pt} % Remove footer underlines
\setlength{\headheight}{13.6pt} % Customize the height of the header

\numberwithin{equation}{section} % Number equations within sections (i.e. 1.1, 1.2, 2.1, 2.2 instead of 1, 2, 3, 4)
\numberwithin{figure}{section} % Number figures within sections (i.e. 1.1, 1.2, 2.1, 2.2 instead of 1, 2, 3, 4)
\numberwithin{table}{section} % Number tables within sections (i.e. 1.1, 1.2, 2.1, 2.2 instead of 1, 2, 3, 4)

\setlength\parindent{0pt} % Removes all indentation from paragraphs - comment this line for an assignment with lots of text

%----------------------------------------------------------------------------------------
%	TITLE SECTION
%----------------------------------------------------------------------------------------

\newcommand{\horrule}[1]{\rule{\linewidth}{#1}} % Create horizontal rule command with 1 argument of height

\title{	
\normalfont \normalsize 
\textsc{Zhiyuan College, Shanghai Jiaotong University} \\ % Your university, school and/or department name(s)
\horrule{0.5pt} \\[0.4cm] % Thin top horizontal rule
\huge CS225: Probability and Computing Homework 2 \\ % The assignment title
\horrule{2pt} \\ % Thick bottom horizontal rule
}

\author{Zihao Ye} % Your name

\date{\normalsize\today} % Today's date or a custom date

\begin{document}

\maketitle % Print the title
\section*{Problem 1}
\begin{enumerate}[(a)]
	\item
    $$\exp\left(-\Psi_X^*(t)\right) = \exp\left(\inf_{\lambda \geq 0}\left(-\lambda t + \Psi_X(\lambda)\right) \right) = \min_{\lambda \geq 0}\frac{E[e^{\lambda|X|}]}{e^{\lambda t}} $$
    此式即\textit{Chernoff Bound}的形式.
    
    由于$\Psi_X(\lambda)$在$\lambda > 0$时为严格凸函数, 因此二阶导数大于零, 因此$(\lambda t - \Psi_X(\lambda))$的极值在$t - \Psi_X'(\lambda) = 0$时取到, 即$\Psi_X'(\lambda) = t$.
    \item
    注意到特征函数$\phi_X(\lambda)$与$\Psi_X(\lambda)$之间的关系:
    $$\Psi_X(\lambda) = \ln\left(\phi_X(-i\lambda)\right) $$
    
    当$X\sim N(\mu, \sigma)$时, $\phi_X(t) = e^{it\mu -\frac{1}{2}\sigma^2t^2}$, 由此可得$$\Psi_X(\lambda) = \ln\left(e^{\mu\lambda + \frac{1}{2}\sigma^2\lambda^2}\right)=\mu\lambda + \frac{1}{2}\sigma^2\lambda^2$$
    
    $ \Psi_X'(\lambda) = \mu + \lambda\sigma^2 = t$即$\lambda = \frac{t - \mu}{\sigma^2}$时$\Psi_X^*(t)$取到极值$\frac{(t-\mu)^2}{2\sigma^2}$(注意到$\lambda$需要非负, 因此该不等式仅在$t \geq \mu$时成立).
    
    $$Pr\{X\geq t\} \leq \exp\left\{-\frac{(t-\mu)^2}{2\sigma^2}\right\}$$	
    
    当$t < \mu$时, $Pr\{X \geq t\} \leq 1$.
    \item
    $$\Psi_X(\lambda) = \nu\left(e^\lambda - 1\right) $$
    
    $\Psi_X'(\lambda) = \nu \exp\{\lambda\} = t$即$\lambda = \ln t - \ln \nu$时$\Psi_X^*(t)$取到极值.
    $$Pr\{X\geq t\} \leq \exp\left\{t - \nu\right\}\nu^t / t^t $$
    \item
    $$\Psi_X(\lambda) = 1-p+pe^{\lambda} $$

	$\Psi_X'(\lambda) = t$即$\lambda = \ln\left(\frac{t(1-p)}{p(1-t)}\right)$时$\Psi_X^*(t)$取得极值.
    
    $$\Psi_X(\lambda) = (1-t)\ln\frac{1-t}{1-p} + t\ln \frac{t}{p} $$
    \item
    由于$X = \sum_{i=1}^{n}X_i$, 且$X_i$独立同分布.
    $$\Psi_X(\lambda) = \ln E[e^{\lambda X}] = \ln \prod_{i=1}^{n}E[e^{\lambda X_i}] = \sum_{i=1}^{n}\Psi_{X_i}(\lambda) = n\Psi_{X_i}(\lambda)$$
    固定$t$, 设$\lambda t - \Psi_X(\lambda)$在$\lambda^*$取极值, 即$\Psi_X'(\lambda^*)=t$, 则$\Psi_X^*(t) = \lambda^*t-\Psi_X(\lambda^*)$,由上式可得$$\sum_{i=1}^{n}\Psi'_{X_i}(\lambda^*)=t, \Psi'_{X_i}(\lambda^*)=\frac{t}{n}$$
    
    因此$\lambda\frac{t}{n} - \Psi_{X_i}(\lambda)$在$\lambda^*$取极值, 即
    $$\Psi_{X_i}^*\left(\frac{t}{n}\right) = \lambda^*\frac{t}{n} - \Psi_{X_i}(\lambda^*)$$
    
    得出$\Psi_{X}(t) = n\Psi_{X_i}^*\left(\frac{t}{n}\right)$.
    
    由上一问的结论, 当$X\sim \mathit{Bin}(n,p)$时:
    
    $$\Psi_{X}^*(t) = n\left((1-t)\ln\frac{1-t}{1-p} + t\ln \frac{t}{p}\right) $$
    
    $$Pr\left\{X\geq t\right\} \leq \left(\left(\frac{1-t}{1-p}\right)^{1-t}\left(\frac{t}{p}\right)^{t}\right)^{-n} $$
    
    当$X_i \sim \mathit{Ge}(1, p)$时:
    $$\Psi_{X_i}(t)=\log\left(\frac{p}{1-qe^{\lambda}}\right)$$
    
    $$\Psi_{X_i}^*(t)=t\ln\frac{t}{q(t+1)} - \ln p(t+1) $$
    则$\Psi_{X}^*(t) = n\Psi_{X_i}^*(\frac{t}{n}) = t\ln\frac{t}{q(t+n)} - n\ln\frac{p(t+n)}{n}$.
    
    $$Pr\left\{X\geq t\right\}\leq \frac{p^n(1-p)^t(t+n)^{t+n}}{n^nt^t} $$
\end{enumerate}
\section*{Problem 2}
\begin{enumerate}[(a)]
	\item
	$$ Pr[|X|\geq \delta] \leq \min_{t\geq 0} \frac{E[e^{t|X|}]}{e^{t \delta}} = \min_{t\geq 0} \frac{\sum\limits_{k=0}^{\infty}\frac{t^k}{k!}E[|X|^k]}{\sum\limits_{k=0}^{\infty}\frac{t^k}{k!}\delta^k} $$
	由于$X$固定, 设$$t^*(\delta)=\arg_{t}\left\{\frac{\sum\limits_{k=0}^{\infty}\frac{t^k}{k!}E[|X|^k]}{\sum\limits_{k=0}^{\infty}\frac{t^k}{k!}\delta^k}\right\}$$
    
    为方便起见, 用$t^*$简写(下同).

	当$t^*\not=0$时, 设$$c = \min\limits_{t\geq 0}\frac{E[e^{t|X|}]}{e^{t\delta}}$$ $$a_k = \frac{{t^*}^k}{k!}E[|X|^k]$$ $$b_k=\frac{{t^*}^k}{k!}\delta^k$$
    
    则$\{a_k\},\{b_k\}$均为正项级数, $c = \frac{\sum\limits_{k=0}^{\infty}a_k}{\sum\limits_{k=0}^{\infty}b_k}$.
    
    若$a_k > cb_k$恒成立, 则有$\sum_{k=0}^{\infty}a_k > \sum_{k=0}^{\infty}cb_k$, 得
    $$c < \frac{\sum\limits_{k=0}^{\infty}a_k}{\sum\limits_{k=0}^{\infty}b_k}$$
    
    矛盾, 因此必存在$k$使得$\frac{e[|X|^k]}{\delta^k} = \frac{a_k}{b_k}\leq c$, 对应的$k$阶矩比\textit{Chernoff Bound}更优.
    
    当$t^*=0$时, $c=1$, 这个情况是trivial的.
    \item
    一般情况下$E[e^{t|X|}]$比$E[|X|^k]$计算难度较低.而且不需要去具体分析应该使用几阶矩, 通用性更好.
\end{enumerate}
\section*{Problem 3}
\begin{enumerate}[(a)]
	\item
    容易证明$(1-p)^n = (1 - \frac{1}{n})^n \geq e^{-1}$($n=1$是个special case,例外).
    $$Pr\left\{R\cap S = \emptyset, R\cap T \not= \emptyset\right\} = (1-p)^n\left(1 - (1-p)^n\right) \geq e^{-1} - e^{-2}$$
    \item
    用$x_1, x_2, \cdots, x_n$表示$s_1, s_2, \cdots, s_n$是否被选, $y_1, y_2, \cdots, y_n$表示$t_1, t_2, \cdots, t_n$是否被选.
    
    令$X = \sum_{i=1}^{n}x_n$, $Y =\sum_{i=1}^{n}y_n$, 由于$x_i, y_i$满足pairwise independent, 可以证明, $E[XY] = E[X]E[Y]$.
    
    $$Pr\left\{X > 0 \wedge Y > 0\right\} =  Pr\left\{XY \geq 1\right\} \leq E[XY] = E[X]E[Y] = p^2n^2$$
   	
    易证$E[X]=np$, $E[X^2] = (np)^2 + np(1-p)$.
    
    一个可以估计$Pr\{X>0\}$的优美方法是使用\textit{Cauchy Inequality}(via 游宇榕):
    $$E[X^2]Pr\{X>0\} \geq (E[X])^2 \Longrightarrow Pr\{X>0\} \geq \frac{np}{1-p+np} $$
    
    原题要求的是
    \begin{eqnarray*}
    Pr\{X=0 \wedge Y > 0\} & = & Pr\{Y>0\} - Pr\{X > 0 \wedge Y > 0\} \\ 
    & \geq & \frac{np}{1-p+np} - (np)^2 \\
    & \geq & \frac{np}{1+np} - (np)^2
    \end{eqnarray*}
    
    当$np=0.3$时, 上式取max$\approx0.14$为一个常数.
    
    因此只需要取$p = 0.3/n$即可.
    
    \textbf{Note:} 另一种估计$Pr\{X>0\}$的方法是尽量糅合X的一阶矩与二阶矩(via 刘志健), $$E[X] = \sum_{i=1}^{k}i Pr\left\{X=i\right\} + \sum_{i=1}^{k}\frac{i^2}{i} Pr\left\{X=i\right\} \leq k Pr\{X > 0\} + E[X^2]k^{-1} $$
\end{enumerate}
\section*{Problem 4}
设$|U| = m$, $U$中元素可列为$u_1, u_2, \cdots, u_m$, 则可以写出如下的Linear Program:
   \begin{alignat*}{2}
\mbox{minimize} \quad & \sum_{j=1}^m{c_j} & {} & {}\\
\mbox{subject to} \quad
& 0\leq c_j \leq 1 & \quad & {\forall j \in [m]}\\
& \sum_{j=1}^{m} M_{ij}c_j \geq 1 & \quad& {\forall i \in [n]}
\end{alignat*}

	容易证明, 原问题的解一定满足上述LP的约束, 因此

	按照如下算法进行rounding, 产生$km$(其中k是个待定的数)个mutually independent的均匀分布在$[0,1]$中的随机变量$\left\{x_{ij}\right\}$. $u_j$被选当且仅当存在某个$i$, 使得在第$i$个trial中$x_{ij}\leq c_j$. 设一共选了$T$个元素, 第$i$个trial选择了$T_i$个元素.

	$$E[T] = E\left[\sum_{i=1}^{k}T_i\right] = \sum_{i=1}^{k}E[T_i] = k\left(\sum_{j=1}^{m}c_j\right)$$
    
    因此这个算法得到解的期望$E[T] = k\cdot OPT$.
    
    下面计算这个算法得到的解合法的概率:
    \begin{eqnarray*}
    Pr\left\{\textrm{所有集合均被touch}\right\} & = &1 - Pr\left\{\textrm{存在未被touch的集合}\right\}\\ 
    & \geq & 1 - \sum_{i=1}^{n} Pr\left\{\textrm{集合$i$未被touch}\right\} \\
    & = & 1 - \sum_{i=1}^{n} \prod_{j=1}^{m}\left(1 - M_{ij}c_j\right) \\
    & \geq & 1 - \sum_{i=1}^{n} e^{-\left(\sum\limits_{j=1}^{m}M_{ij}c_j\right)} \\
    & \geq & 1 - n\cdot e^{-k}
    \end{eqnarray*}
    
    令$k = 2\log(n)$, 则$1 - n\cdot e^{-k} = 1 - \frac{1}{n}$, 此时算法w.h.p.将得到一个原问题的$2\log(n)=O(\log(n))$的近似.
    
\section*{Problem 5	}
以下两问均使用了$x_1, x_2, \cdots, x_n$mutually independent的条件, 而这个条件在原题中并未给出.
\begin{enumerate}[(a)]
	\item
	用$I_i$表示从第$i$为从第$i$位开始到$i+k-1$位全部为$1$, 第$i+k$位为$0$(当$i + k > 0$时除外)的indicator.
	则$X_k = \sum\limits_{i=1}^{n - k} I_i + I_{n-k+1}$, 由此得出:
	$$E[X_k] = \sum_{i = 1}^{n - k}E[I-i] + I_{n-k+1} = (n - k) 2^{-(k+1)} + 2^{-k} = (n - k + 2) 2^{-(k+1)}$$
    \item
    设$f(x_1, x_2, \cdots, x_n)$为$x_1, x_2, \cdots x_n$组成的长度大于$k$的turn个数, 则有
    $$|f(x_1, x_2, \cdots, x_i \cdots, x_n) - f(x_1, x_2, \cdots, y_i, \cdots, x_n)| \leq 1 $$
    满足\textit{Lipschitz Condition}, 则有:
    $$Pr\left\{|X_k - E[X_k]|\geq t\right\}=Pr\left\{\left|f(\mathbf{X}) - E[f(\mathbf{X})]\right| \geq t\right\} \leq 2\exp\left(-\frac{t^2}{2n}\right) $$
    因此$t=\sqrt[]{2n\log n}$是一个\textit{concentration bound}, 此时$Pr\left\{|X_k - E[X_k]|\geq t\right\}=O(\frac{1}{n})$.
	
\end{enumerate}
\section*{Problem 6}
\begin{enumerate}[(a)]
\item
令$d_H^{(i)}(X, A)$为$E[d_H(X, A)\mid X_1, X_2, \cdots, X_n]$, 则$\{d_H^{(i)}(X,A)\}$是一个\textit{Doob sequence}.

设$Y_i = d_H^{(i)}(X, A)$, 则易证:
$$|Y_i - Y_{i-1}| \leq 1$$

由\textit{Azuma's Inequality}, 有
$$Pr\left\{Y_n - Y_0\geq -t\right\}\leq \exp\left\{-\frac{t^2}{2n}\right\} $$

注意到$Y_0 = E[d_H(X,A)]$, $Y_n = d_H(X,A)$, 
$$\mu(A) = Pr\left\{Y_n = 0\right\} \leq Pr\left\{Y_n - Y_0\geq -Y_0\right\} \leq \exp\left\{-\frac{Y_0^2}{2n}\right\}$$

解得$Y_0\leq \sqrt{2n\ln{\frac{1}{\mu(A)}}}$.
\item
原问题即证明$\forall \epsilon, \exists \delta > 0, \mu(A)\geq2^{-\delta n} \Longrightarrow \mu(A_{\epsilon n}) > 1 - 2^{-\delta n} $

由\textit{Azuma's Inequality}易证明:
\begin{eqnarray*}
\mu(A_{\epsilon n}) &=& Pr\left\{d_H(X,A) \leq \epsilon n\right\}\\ & = &Pr\left\{d_H(X, A) - E[d_H(X,A)] \leq \epsilon n - E[d_H(X,A)]\right\}\\ & \geq & 1 - \exp\left\{-\frac{\left(\epsilon n - E[d_H(X,A)]\right)^2}{2n}\right\}
\end{eqnarray*}
令$0 < \delta < \frac{\epsilon^2}{8\ln(2)}$, 则$$\epsilon n - E[d_H(X,A)] \geq \epsilon n - \sqrt{2n\ln\frac{1}{\mu(A)}} \geq \left(\epsilon - \sqrt{2\delta \ln(2)}\right)n > \frac{\epsilon}{2}n > 0$$.

由此:
$$1 - \exp\left\{-\frac{\left(\epsilon n - E[d_H(X,A)]\right)^2}{2n}\right\} > 1 - \exp\left\{-\frac{\epsilon^2 n}{8} \right\}$$

而$$1 - 2^{-\delta n} < 1 - 2^{-\frac{\epsilon^2}{8 \ln (2)}n} = 1 - \exp\left\{-\frac{\epsilon^2 n}{8}\right\} $$

由此任取$\delta \in \left(0, \frac{\epsilon^2}{8\ln(2)}\right)$即可.
\end{enumerate}

\section*{Acknowledgement}
感谢贝小辉老师在近似算法课上介绍了Problem 4的randomized rounding的做法.

感谢游宇榕和刘志健同学在第3题中给予的启发.

\end{document}